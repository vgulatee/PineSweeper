\documentclass[12pt, titlepage]{article}
\usepackage{fullpage}
\usepackage[round]{natbib}
\usepackage{multirow}
\usepackage{booktabs}
\usepackage{tabularx}
\usepackage{graphicx}
\usepackage{float}
\usepackage{hyperref}
\hypersetup{
    colorlinks,
    citecolor=black,
    filecolor=black,
    linkcolor=red,
    urlcolor=blue
}
\usepackage[round]{natbib}
\newcounter{acnum}
\newcommand{\actheacnum}{AC\theacnum}
\newcommand{\acref}[1]{AC\ref{#1}}
\newcounter{ucnum}
\newcommand{\uctheucnum}{UC\theucnum}
\newcommand{\uref}[1]{UC\ref{#1}}
\newcounter{mnum}
\newcommand{\mthemnum}{M\themnum}
\newcommand{\mref}[1]{M\ref{#1}}
\title{SE 3XA3: Software Requirements Specification\\PineSweeper}
\author{Team 7, PineApple
		\\ Prince Sandhu; sandhps2
		\\ Varun Rathore; rathorvs
		\\ Vishesh Gulatee; gulatev
}
\date{\today}
\begin{document}
\maketitle
\pagenumbering{roman}
\tableofcontents
\listoftables
\listoffigures
\clearpage
\begin{table}[!htbp]
\caption{\bf Revision History}
\begin{tabularx}{\textwidth}{p{3cm}p{3cm}X}
\toprule {\bf Date} & {\bf Version} & {\bf Notes}\\
\midrule
November 08 & 1.0 & Created Initial LaTex File\\
November 08 & 1.1 & Anticipated and Unlikely Changes Initial\\
November 10 & 1.2 & Module Hierarchy Initial\\
November 10 & 1.3 & Module Decomposition Initial\\
November 11 & 1.4 & Traceability Matrix Initial\\
November 12 & 2.0 & Anticipated and Unlikely Changes Final\\
November 12 & 2.1 & Module Hierarchy Final\\
November 12 & 2.2 & Module Decomposition Final\\
November 13 & 2.3 & Traceability Matrix Final\\
\bottomrule
\end{tabularx}
\end{table}
\pagenumbering{arabic}

\newpage
\section{Introduction}
Decomposing a system into modules is a commonly accepted approach to developing
software.  A module is a work assignment for a programmer or programming
team~\citep{ParnasEtAl1984}.  We advocate a decomposition
based on the principle of information hiding~\citep{Parnas1972a}.  This
principle supports design for change, because the ``secrets'' that each module
hides represent likely future changes.  Design for change is valuable in SC,
where modifications are frequent, especially during initial development as the
solution space is explored.  
Our design follows the rules layed out by \citet{ParnasEtAl1984}, as follows:

\begin{itemize}
\item System details that are likely to change independently should be the
  secrets of separate modules.
\item Each data structure is used in only one module.
\item Any other program that requires information stored in a module's data
  structures must obtain it by calling access programs belonging to that module.
\end{itemize}

After completing the first stage of the design, the Software Requirements
Specification (SRS), the Module Guide (MG) is developed~\citep{ParnasEtAl1984}. The MG
specifies the modular structure of the system and is intended to allow both
designers and maintainers to easily identify the parts of the software.  The
potential readers of this document are as follows:

\begin{itemize}
\item New project members: This document can be a guide for a new project member
  to easily understand the overall structure and quickly find the
  relevant modules they are searching for.
\item Maintainers: The hierarchical structure of the module guide improves the
  maintainers' understanding when they need to make changes to the system. It is
  important for a maintainer to update the relevant sections of the document
  after changes have been made.
\item Designers: Once the module guide has been written, it can be used to
  check for consistency, feasibility and flexibility. Designers can verify the
  system in various ways, such as consistency among modules, feasibility of the
  decomposition, and flexibility of the design.
\end{itemize}

The rest of the document is organized as follows. Section
\ref{SecChange} lists the anticipated and unlikely changes of the software
requirements. Section \ref{SecMH} summarizes the module decomposition that
was constructed according to the likely changes. Section \ref{SecConnection}
specifies the connections between the software requirements and the
modules. Section \ref{SecMD} gives a detailed description of the
modules. Section \ref{SecTM} includes two traceability matrices. One checks
the completeness of the design against the requirements provided in the SRS. The
other shows the relation between anticipated changes and the modules. Section
\ref{SecUse} describes the use relation between modules.

\section{Anticipated and Unlikely Changes} \label{SecChange}
This section lists possible changes to the system. According to the likeliness
of the change, the possible changes are classified into two
categories. Anticipated changes are listed in Section \ref{SecAchange}, and
unlikely changes are listed in Section \ref{SecUchange}.

\subsection{Anticipated Changes} \label{SecAchange}
Anticipated changes are the source of the information that is to be hidden
inside the modules. Ideally, changing one of the anticipated changes will only
require changing the one module that hides the associated decision. The approach
adapted here is called design for change.

\begin{description}
\item[\refstepcounter{acnum} \actheacnum \label{acHardware}:] The specific
  hardware on which the software is running.
\item[\refstepcounter{acnum} \actheacnum \label{acInput}:] The format of the
  initial input data.
\item[\refstepcounter{acnum} \actheacnum \label{acThemes}:] Different
  visual themes.
\item[\refstepcounter{acnum} \actheacnum \label{acShapes}:] Different
  shape of the mine grid.
\end{description}

\subsection{Unlikely Changes} \label{SecUchange}
The module design should be as general as possible. However, a general system is
more complex. Sometimes this complexity is not necessary. Fixing some design
decisions at the system architecture stage can simplify the software design. If
these decision should later need to be changed, then many parts of the design
will potentially need to be modified. Hence, it is not intended that these
decisions will be changed.

\begin{description}
\item[\refstepcounter{ucnum} \uctheucnum \label{ucIO}:] Input/Output devices
  (Input: File and/or Keyboard, Output: File, Memory, and/or Screen).
\item[\refstepcounter{ucnum} \uctheucnum \label{ucInput}:] There will always be
  a source of input data external to the software.
\item[\refstepcounter{ucnum} \uctheucnum \label{ucButton}:] Clicking on a button
  will always uncover the the game item that is situated under the button.
\item[\refstepcounter{ucnum} \uctheucnum \label{ucTermination}:] Clicking on a
  \textit{Pineapple} will always result in the game state being terminated.
\item[\refstepcounter{ucnum} \uctheucnum \label{ucReset}:] Clicking on the
  reset button will always reset the game.
\item[\refstepcounter{ucnum} \uctheucnum \label{ucGrid}:] The game will always
  be played on a grid.
\end{description}

\newpage
\section{Module Hierarchy} \label{SecMH}
This section provides an overview of the module design. Modules are summarized
in a hierarchy decomposed by secrets in Table \ref{TblMH}. The modules listed
below, which are leaves in the hierarchy tree, are the modules that will
actually be implemented.

\begin{description}
\item [\refstepcounter{mnum} \mthemnum \label{mIF}:] Input Format Module
\item [\refstepcounter{mnum} \mthemnum \label{mGMS}:] Game Mode Selection Module
\item [\refstepcounter{mnum} \mthemnum \label{mPS}:] Preferences Selection Module
\item [\refstepcounter{mnum} \mthemnum \label{mSD}:] Shared Data Module
\item [\refstepcounter{mnum} \mthemnum \label{mSU}:] State Update Module
\item [\refstepcounter{mnum} \mthemnum \label{mO}:] Output Module
\item [\refstepcounter{mnum} \mthemnum \label{mCDS}:] Cell Data Structure Module
\item [\refstepcounter{mnum} \mthemnum \label{mGG}:] Grid Generation Module
\item [\refstepcounter{mnum} \mthemnum \label{mGT}:] Grid Traversal Module
\item [\refstepcounter{mnum} \mthemnum \label{mAO}:] Application Objective Module
\item [\refstepcounter{mnum} \mthemnum \label{mHH}:] Hardware Hiding Module
\end{description}

\begin{table}[h!]
\centering
\begin{tabular}{p{0.3\textwidth} p{0.6\textwidth}}
\toprule
\textbf{Level 1} & \textbf{Level 2}\\
\midrule
{Hardware-Hiding Module} & ~ \\
\midrule
\multirow{7}{0.3\textwidth}{Behaviour-Hiding Module} & Input Format Module\\
& Game Mode Selection Module\\
& Preferences Selection Module\\
& Shared Data Module\\
& State Update Module\\
& Output Module\\
\midrule
\multirow{3}{0.3\textwidth}{Software Decision Module} & {Cell Data Structure Module}\\
& Grid Generation Module\\
& Grid Traversal Module\\
& Application Objective Module\\
\bottomrule
\end{tabular}
\caption{Module Hierarchy}
\label{TblMH}
\end{table}

\section{Connection Between Requirements and Design} \label{SecConnection}
The design of the system is intended to satisfy the requirements developed in
the SRS. In this stage, the system is decomposed into modules. The connection
between requirements and modules is listed in Table \ref{TblRT}.

\section{Module Decomposition} \label{SecMD}
Modules are decomposed according to the principle of ``information hiding''
proposed by \citet{ParnasEtAl1984}. The \emph{Secrets} field in a module
decomposition is a brief statement of the design decision hidden by the
module. The \emph{Services} field specifies \emph{what} the module will do
without documenting \emph{how} to do it. For each module, a suggestion for the
implementing software is given under the \emph{Implemented By} title. If the
entry is \emph{OS}, this means that the module is provided by the operating
system or by standard programming language libraries.  Also indicate if the
module will be implemented specifically for the software.
Only the leaf modules in the hierarchy have to be implemented.
If a dash (\emph{--}) is shown, this means that the module is not a leaf and
will not have to be implemented. Whether or not this module is implemented
depends on the programming language selected.

\subsection{Hardware Hiding Modules (\mref{mHH})}
\begin{description}
\item[Secrets:]The data structure and algorithm used to implement the virtual
  hardware.
\item[Services:]Serves as a virtual hardware used by the rest of the
  system. This module provides the interface between the hardware and the
  software. So, the system can use it to display outputs or to accept inputs.
\item[Implemented By:] OS and Java Virtual Machine
\end{description}

\subsection{Behaviour-Hiding Module}
\begin{description}
\item[Secrets:]The contents of the required behaviours.
\item[Services:]Includes programs that provide externally visible behaviour of
  the system as specified in the software requirements specification (SRS)
  documents. This module serves as a communication layer between the
  hardware-hiding module and the software decision module. The programs in this
  module will need to change if there are changes in the SRS.
\item[Implemented By:] --
\end{description}

\subsubsection{Input Format Module (\mref{mIF})}
\begin{description}
\item[Secrets:]The format and structure of the input data.
\item[Services:]Converts the input data into the data structure used by the
  input parameters module.
\item[Implemented By:] \textit{PineSweeper}\\
\end{description}

\subsubsection{Game Mode Selection Module (\mref{mGMS})}
\begin{description}
\item[Secrets:] textcolor{red}{Grid and Window Size.}
\item[Services:] Allows user to select the difficulty of the game that they
  wish to play.
\item[Implemented By:] \textit{PineSweeper}\\
\end{description}

\subsubsection{Preferences Selection Module (\mref{mPS})}
\begin{description}
\item[Secrets:] User preference.
\item[Services:] Allows user to select the visual theme (provided),
   for the application to embody.
\item[Implemented By:] \textit{Java Libraries and OS}\\
\end{description}

\subsubsection{Shared Data Module (\mref{mSD})}
\begin{description}
\item[Secrets:] \textcolor{red}{Difficulty level and theme choice.}
\item[Services:] Stores and shares resources needed across all
   modules.
\item[Implemented By:] \textit{PineSweeper and Java Libraries}\\
\end{description}

\subsubsection{State Update Module (\mref{mSU})}
\begin{description}
\item[Secrets:] \textcolor{red}{Grid data.}
\item[Services:] Updates and manipulates state variables and
   methods based on user response. Provides A.I. of the application.
\item[Implemented By:] \textit{PineSweeper}\\
\end{description}

\subsubsection{Output Module (\mref{mO})}
\begin{description}
\item[Secrets:] \textcolor{red}{Application window.}
\item[Services:] Updates application response to user input,
   by providing a graphical representation of the game state.
\item[Implemented By:] \textit{PineSweeper and Java Libraries}\\
\end{description}

\subsubsection{\textcolor{red}{Application Objective Module} (\mref{mAO})}
\begin{description}
\item[Secrets:] Model
\item[Services:] \item[Implemented By:] \textit{PineSweeper}\\
\end{description}

\newpage
\subsection{Software Decision Module}
\begin{description}
\item[Secrets:] The design decision based on mathematical theorems, physical
  facts, or programming considerations. The secrets of this module are
  \emph{not} described in the SRS.
\item[Services:] Includes data structure and algorithms used in the system that
  do not provide direct interaction with the user. 
  % Changes in these modules are more likely to be motivated by a desire to
  % improve performance than by externally imposed changes.
\item[Implemented By:] -- \\
\end{description}

\subsubsection{Cell Data Structure Module (\mref{mCDS})}
\begin{description}
\item[Secrets:] Cells
\item[Services:] Stores Cell objects, that represent the buttons of the
   Grid in the application.
\item[Implemented By:] \textit{PineSweeper and Java Libraries}\\
\end{description}

\subsubsection{\textcolor{red}{Grid Generation Module (\mref{mGG})}}
\begin{description}
\item[Secrets:] Grid
\item[Services:] Algorithm that describes generation of grid 
   containing cells and its contents.
\item[Implemented By:] \textit{PineSweeper}\\
\end{description}

\subsubsection{\textcolor{red}{Grid Traversal Module} (\mref{mGT})}
\begin{description}
\item[Secrets:] Grid
\item[Services:] Graph search algorithm that describes the 
   nature with which the grid is traversed by the application.
\item[Implemented By:] \textit{PineSweeper}\\
\end{description}

\newpage
\section{Traceability Matrix} \label{SecTM}
This section shows two traceability matrices: between the modules and the
requirements and between the modules and the anticipated changes.
% the table should use mref, the requirements should be named, use something
% like fref
\begin{table}[H]
\centering
\begin{tabular}{p{0.2\textwidth} p{0.6\textwidth}}
\toprule
\textbf{Req.} & \textbf{Modules}\\
\midrule
R1 & \mref{mGMS}, \mref{mSU}\\
R2 & \mref{mGG}, \mref{mCDS}\\
R3 & \mref{mSU}, \mref{mO}\\
R4 & \mref{mGT}, \mref{mGG}\\
R5 & \mref{mSU}, \mref{mAO}\\
R6 & \mref{mGG}, \mref{mGT}\\
R7 & \mref{mSU}, \mref{mO}\\
R8 & \mref{mSU}, \mref{mO}\\
R9 & \mref{mSD}, \mref{mSU}, \mref{mO}\\
R10 & \mref{mO}\\
R11 & \mref{mHH}, \mref{mO}\\
R12 & \mref{mIF}, \mref{mGMS}, \mref{mO}, \mref{mGG}\\
R13 & \mref{mGMS}, \mref{mPS}\\
R14 & \mref{mPS}, \mref{mO}, \mref{mHH}\\
R15 & \mref{mSU}, \mref{mO}, \mref{mHH}\\
R16 & \mref{mHH}\\
R17 & \mref{mSU}, \mref{mO}, \mref{mHH}\\
R18 & \mref{mHH}\\
R19 & \mref{mHH}\\
R20 & \mref{mSU}, \mref{mO}, \mref{mGG}\\
\bottomrule
\end{tabular}
\caption{Trace Between Requirements and Modules}
\label{TblRT}
\end{table}

\begin{table}[H]
\centering
\begin{tabular}{p{0.2\textwidth} p{0.6\textwidth}}
\toprule
\textbf{AC} & \textbf{Modules}\\
\midrule
\acref{acHardware} & \mref{mHH}\\
\acref{acInput} & \mref{mIF}\\
\acref{acThemes} & \mref{mO}\\
\acref{acShapes} & \mref{mO}\\
\end{tabular}
\caption{Trace Between Anticipated Changes and Modules}
\label{TblACT}
\end{table}

\newpage
%Included uses hierarchy image.
\section{Use Hierarchy Between Modules} \label{SecUse}
In this section, the uses hierarchy between modules is
provided. \citet{Parnas1978} said of two programs A and B that A {\em uses} B if
correct execution of B may be necessary for A to complete the task described in
its specification. That is, A {\em uses} B if there exist situations in which
the correct functioning of A depends upon the availability of a correct
implementation of B.  Figure \ref{FigUH} illustrates the use relation between
the modules. It can be seen that the graph is a directed acyclic graph
(DAG). Each level of the hierarchy offers a testable and usable subset of the
system, and modules in the higher level of the hierarchy are essentially simpler
because they use modules from the lower levels.
\begin{figure}[H]
\centering
\includegraphics[width=0.7\textwidth]{UsesHierarchy.png}
\caption{\textcolor{red}{Use hierarchy among modules}}
\label{FigUH}
\end{figure}
%\section*{References}
\bibliographystyle {plainnat}
\bibliography {MG}
\end{document}