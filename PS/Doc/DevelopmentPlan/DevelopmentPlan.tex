\documentclass{article}
\usepackage{hyperref}
\usepackage{booktabs}
\usepackage{tabularx}
\usepackage{color}
\title{SE 3XA3: Development Plan\\PineSweeper}
\author{Team 07, PineApple
		\\ Prince Sandhu; sandhps2
		\\ Varun Rathore; rathorvs
		\\ Vishesh Gulatee; gulatev
}

\date{September 30, 2016}
\begin{document}
\begin{table}[hp]
\caption{Revision History} \label{TblRevisionHistory}
\begin{tabularx}{\textwidth}{llX}
\toprule
\textbf{Date} & \textbf{Developer(s)} & \textbf{Change}\\
\midrule
23/09/2016 & P.S., V.R., V.G. & Initial Drafts of Communication, Roles\\
27/09/2016 & P.S., V.R., V.G. & Initial Drafts of Meeting Plan, Technology, Coding Style\\
29/09/2016 & P.S., V.R., V.G. & Final Drafts of all sections\\
30/09/2016 & P.S., V.R., V.G. & Final Draft of Revision 0\\
03/12/2016 & P.S., V.R., V.G. & Revision 1\\
... & ... & ...\\
\bottomrule
\end{tabularx}
\end{table}

\newpage
\maketitle

\section{\textcolor{red}{Abstract}}
%Added the 'Abstract' section heading.
Using the intrinsic software engineering design principles, the fundamental mission of \textit{PineApple} is to implement \textit{PineSweeper},
 a software gaming application that not only individuals can utilize to entertain themselves, but also challenge their mental faculties. 
\textit{PineSweeper} will be a replication of the traditional game of MineSweeper, which can be found installed on almost every computer.
The following document comprises of \textit{PineApple}'s plan to ensure the successful development of the application.

\section{Team Meeting Plan}
%Assigned the role of scribe and discussed roles more in depth.
The meetings are to be performed either in the lab, a library on campus of \textit{McMaster University} or via \textit{Google Hangouts}. During
the lab, the meetings are to take place within the allotted two hours, as it also provides \textit{PineApple} the time to work on the program and 
documentation together. The target frequency is four to five meetings per week, with two meetings held in the lab and the rest outside of lab
time. \textcolor{red}{Regarding the roles, the chair and scribe positions are to alternate amongst the team members with each meeting. The
appointed chair is assigned the task of leading the meeting, whereas the scribe is to record the decisions made during the meeting. Lastly,
Dr. Spencer Smith and the teaching assistant, Christopher McDonald, are the communicators; any issues that arise while the project is in
progress can be addressed to them.}

\section{Team Communication Plan}

Communication outside of campus and the meetings will be accomplished using various online tools. To schedule meetings, discuss project
progress, and minor issues, \textit{PineApple} will be using a group chat via Facebook Messenger. Queries and issues specific to the
codebase of the application and its feature will most likely be raised, discussed and resolved using Gitlab's issues tracking functionality, as
well as using pull requests when implementing feature branches. In addition, every member has exchanged their contact information,
including their cell phone numbers, so they can still be contacted if they do not have access to the internet. 

\section{Team Member Roles}

Upon collective agreement, there is no group leader of \textit{PineApple}, instead each member is responsible for supervising certain
aspects of the design procedure. As such, the team member roles are as follows: Prince Sandhu plays the role of the programmer, covers
the documentation, and is the illustrator. Varun Singh Rathore is also a programmer, and plays a significant role in testing. Vishesh Gulatee
plays the role of the programmer, and specializes in Git. If there is a conflict between two team members, the third member is to take a
neutral stance and serve as a mediator.

\section{Git Workflow Plan}

Git will be implemented in the development workspace using Feature Branch Workflow, which is a variation of the Centralized Workflow
system. The centralized workflow allows each team member/developer to clone a copy of the entire \textit{PineSweeper} project from the
central repository or the default development branch called "master". The local copy is termed the local repository and the additions and
changes made to this repository will be stored locally. These changes will be published to the official project in the central repository via git
commands such as commit and push, updating master branch. The changes/additions can be reviewed or experimented on by other
developers of \textit{PineApple} by "fetching" the updated central commits into their local repositories. Hence, this system allows the team to
manage conflicts between the central and local repositories by rebasing (adding changes on top of  what everyone else has done) by
fetching the project from master and then updating the central repository. Furthermore if local changes directly conflict with upstream
commits, Git provides the developers with the ability to manually resolve conflicts.

\vspace{5mm}
Using dedicated branches (other than main) for all feature development in the workflow allows the team to model the development
workspace after the design architecture used for the \textit{PineSweeper} project (Model, View and Controller Architecture). This
encapsulation allows multiple developers to collaborate on a particular feature, all the while ensuring that codebase in master branch isn?t
disturbed or contains broken code. Using feature branches also allows the team to streamline communication, constricting the focus of the
discussion to a particular feature and problem, at any given time. For more general issues and focuses, GitLab's issue tracking system will
be used. Throughout the development and documentation phase, progress will be marked by the use of tags. This allows the team to
organize key commits for future references. 

\section{Proof of Concept Demonstration Plan}
%Addressed what the application is to do for the Proof of Concept demonstration.
During the proof of concept demonstration, \textcolor{red}{\textit{PineApple} expects to demonstrate a simplistic user interface of the
\textit{PineSweeper} game board, which must be able to reveal what is underneath the covered tile when the tile is clicked upon by the
user. It must also be able to generate \textit{PineMines} and numbers corresponding to the number of mines bordering each cell.} The proof
of concept must be executable in the Windows and Mac operating systems.

\vspace{5mm}
Regarding the risks, the \textit{PineSweeper}'s properties and interface will be examined in order to look at any uncertainties. Some of the
aspects of the game that will be examined include the clicking of the buttons to reveal what is beneath it, and the number of mines in the
game. A significant risk that \textit{PineSweeper} may face in the process includes the unit testing of the Model class for gameplay
validation. On the other hand, any required libraries are not difficult to install, portability is not a significant concern, and the game is not
to be run on a server, so concurrent players and server capacity is not a risk.

\section{Technology}
%Addressed plans for the documentation of the code.
The programming language to be used in the development and implementation of this project is Java, and will be accomplished using the
Eclipse Integrated Development Environment. Generating documents involves a two-step process; first rough drafts and notes using Google
Documents are constructed, followed by generating official versions of the documents in \textit{LaTeX}. The documents are linearly
committed to directories in master branch in Git. \textcolor{red}{The Java code is to be documented using JavaDocs and generated
using \textit{Doxygen}.}

\vspace{5mm}
Testing of the application will involve degrees of internal testing ($\alpha$-testing) and external testing ($\beta$-testing), based on the software development phase. This includes conducting both manual testing and test automation.

\vspace{5mm}
The manual testing phase will be conducted in the early stages of development and implementation, since the program will be undergoing
various changes and updates. It will consist of front-end testing and back-end testing, that will be implemented by a combination of
developers of the codebase (including both programmers who worked on the feature and programmers who did not). The next phase of
manual testing is to include survey testing (or open $\beta$-testing) which will be conducted using unbiased potential users of the
application. This type of testing necessary as it allows for usability testing or testing the human experience and integrating consumer
feedback to the validation process of the software development procedure.

\vspace{5mm}
 Automation testing will include modular testing of each integral feature (especially Model), black box unit testing via /textit{JUnit} and black
 box automated integrated testing (testing the application as a whole), as part of closed $\alpha$ and $\beta$ phases.

\section{Coding Style}
%Addressed the issue of not providing a Coding Style. Link has also been included.
\textcolor{red}{\href{https://google.github.io/styleguide/javaguide.html}{\textit{Google's Java Style Guide}} is to be utilized for this project}. In
addition, \textit{PineSweeper} is to be implemented using the Model, View, and Controller (MVC) architectural framework.

\section{Project Schedule}
%Changed order of packages
\textcolor{red}{Link to \href{run:ProjectSchedule.gan} {\textit{PineApple}'s Gantt Chart}}

\section{Project Review}
%Revision 1
\textcolor{red}{\textit{PineSweeper} has been a successful project over the past few months for \textit{PineApple}, who achieved 
all of the goals detailed before the commencement of the project with a great degree of success. Not only did \textit{PineApple} manage
to successfully replicate the open source project from scratch and make numerous improvements, it managed to do the aforementioned
by adhering to the software design process, which included providing the appropriate formal documentation. The cornerstone project
underscored the iterative characteristics of the software development process. Throughout the course of the project, the team maintained
exceptional communication and learned valuable project and time management skills.}\\

\textcolor{red}{One aspect of the project that did not go exactly as planned was conforming to the project's gantt chart. The gantt chart,
which is a significant project time management tool, serves as a guide for each team member when direction is needed. Even though the
team was ambitious in the gantt chart, it was somewhat difficult meeting most of the internal deadlines, as the deadlines of assignments
from our five other courses played a significant role. Nevertheless, \textit{PineSweeper} remained relatively organized, as the team was
able to meet all deadlines with ease.}\\

\newpage
\textcolor{red}{Regarding the in-person team meetings, conforming to the schedule was challenging at times. The aforementioned is
due to one of the team member being a commuting student. Nevertheless, if certain meetings were required, they would be conducted
on \textit{Google Hangouts or Skype}. \textit{PineApple} was satisfied by the team roles and the team communication and would not
change the plan if it were to embark on another project.}

\end{document}